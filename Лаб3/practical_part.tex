\chapter{Практическая часть}

\section{Задание 1 Написать функцию, которая принимает целое число и возвращает первое
	четное число, не меньшее аргумента.}
\begin{center}
	\begin{lstlisting}[label=a1, caption={Выражение 1}]
(defun t1(x) (if(eq 0 (mod x 2)) x (+ x 1)))
\end{lstlisting}
\end{center}

\section{Задание 2 Написать функцию, которая принимает число и возвращает число
	того же знака, но с модулем на 1 больше модуля аргумента.}
\begin{center}
	\begin{lstlisting}[label=a2, caption={Выражение 2}]
(defun t2(x) (if (> x 0) (+ x 1) (- x 1)))
	\end{lstlisting}
\end{center}

\section{Задание 3 Написать функцию, которая принимает два числа и возвращает
	список из этих чисел, расположенный по возрастанию.}
\begin{center}
	\begin{lstlisting}[label=a3, caption={Выражение 3}]
(defun t3(x y) (if (< x y) (list x y ) (list y x)))
	\end{lstlisting}
\end{center}

\section{Задание 4 Написать функцию, которая принимает три числа и возвращает Т только
	тогда, когда первое число расположено между вторым и третьим.}
\begin{center}
	\begin{lstlisting}[label=a4, caption={Выражение 4}]
(defun t4(x y z) (and (< y x) (< x z)))
	\end{lstlisting}
\end{center}