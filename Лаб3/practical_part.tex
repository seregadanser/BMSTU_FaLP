\chapter{Практическая часть}

\section{Задание 1 Написать функцию, которая принимает целое число и возвращает первое
	четное число, не меньшее аргумента.}
\begin{center}
	\begin{lstlisting}[label=a1, caption={Выражение 1}]
(defun t1(x) (if(eq 0 (mod x 2)) x (+ x 1)))
\end{lstlisting}
\end{center}

\section{Задание 2 Написать функцию, которая принимает число и возвращает число
	того же знака, но с модулем на 1 больше модуля аргумента.}
\begin{center}
	\begin{lstlisting}[label=a2, caption={Выражение 2}]
(defun t2(x) (if (> x 0) (+ x 1) (- x 1)))
	\end{lstlisting}
\end{center}

\section{Задание 3 Написать функцию, которая принимает два числа и возвращает
	список из этих чисел, расположенный по возрастанию.}
\begin{center}
	\begin{lstlisting}[label=a3, caption={Выражение 3}]
(defun t3(x y) (if (< x y) (list x y ) (list y x)))
	\end{lstlisting}
\end{center}

\section{Задание 4 Написать функцию, которая принимает три числа и возвращает Т только
	тогда, когда первое число расположено между вторым и третьим.}
\begin{center}
	\begin{lstlisting}[label=a4, caption={Выражение 4}]
(defun t4(x y z) (and (< y x) (< x z)))
	\end{lstlisting}
\end{center}

\section{Задание 5 Каков результат вычисления следующих выражений?}
\begin{center}
	\begin{lstlisting}[label=a5, caption={Выражение 5}]
(and 'fee 'fie 'foe) ; foe
(or nil 'fie 'foe) ; fie
(and (equal 'abc 'abc) 'yes) ; yes
(or 'fee 'fie 'foe) ; fee
(and nil 'fie 'foe) ; nil
(or (equal 'abc 'abc) 'yes) ; T
	\end{lstlisting}
\end{center}

\section{Задание 6  Написать предикат, который принимает два числа-аргумента и возвращает
	Т, если первое число не меньше второго.}
\begin{center}
	\begin{lstlisting}[label=a6, caption={Выражение 6}]
(defun t6(x y) (>= x y))
	\end{lstlisting}
\end{center}

\section{Задание 7 Какой из следующих двух вариантов предиката ошибочен и почему?}
\begin{center}
	\begin{lstlisting}[label=a7, caption={Выражение 7}]
(defun pred1 (x) (and (numberp x) (plusp x)))
	\end{lstlisting}
\begin{lstlisting}[label=a8, caption={Выражение 8}]
(defun pred2 (x) (and (plusp x)(numberp x)))
\end{lstlisting}
\end{center}

Ошибочным является предикат \ref{a8} из-за отсутствия проверки на число перед увеличением. 

\section{Задание 8 Решить задачу 4, используя для ее решения конструкции:
	только IF, только COND, только AND/OR.}
\begin{center}
	\begin{lstlisting}[label=a9, caption={Выражение 9}]
(defun t8(x y z) (if (< y x) (if (< x z) T Nil) Nil))
	\end{lstlisting}
	\begin{lstlisting}[label=a10, caption={Выражение 10}]
(defun t8(x y z) (cond ((< y x) (cond ((< x z) T) (T Nil))) (T Nil)))
\end{lstlisting}
\end{center}

\section{Задание 9 Переписать функцию how-alike, приведенную в лекции и использующую COND, используя
	только конструкции IF, AND/OR.}
\begin{center}
	\begin{lstlisting}[label=a11, caption={Выражение 11}]
(defun how_alike_if (x y)
(if (or (= x y) (equal x y))
'the_same
(if (and (oddp x) (oddp y))
'both_odd
(if (and (evenp x) (evenp y))
'both_even
'difference))))
	\end{lstlisting}
\end{center}