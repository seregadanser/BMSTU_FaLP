\chapter{Практическая часть}
\section{Задание 1}
Каковы результаты вычисления следующих выражений?
\begin{center}	
	\begin{lstlisting}[label=a, caption={Выражение 1}]
		(cons lstl lst2) ; ((a b c) d e)
		(list lst1 lst2) ; ((a b c) (d e))
		(append lst1 lst2) ; (a b c d e)
	\end{lstlisting}
\end{center}

\section{Задание 2}
Каковы результаты вычисления следующих выражений?
\begin{center}	
	\begin{lstlisting}[label=b, caption={Выражение 2}]
		(reverse '(a b c)) ; (c b a)
		(reverse '(a b (c (d)))) ; ((c (d)) b a)
		(reverse '(a)) ; (a)
		(last '(a b c)) ; (c)
		(last '(a)) ; (a)
		(last '((a b c))) ;((a b c))
		(reverse ()) ; Nil
		(reverse '((a b c))) ; ((a b c))
		(last '(a b (c))) ; ((c))
		(last ()) ; Nil
	\end{lstlisting}
\end{center}

\section{Задание 3}
Написать, по крайней мере, два варианта функции, которая возвращает
последний элемент своего списка-аргумента. 
\begin{center}	
	\begin{lstlisting}[label=c, caption={Выражение 3}]
	(defun get_last (lst)
		(car (last lst))
	)
	
	(defun get_last (lst)
		(if (cdr lst)
		(get_last(cdr lst))
		(car lst)
		)
	)
	\end{lstlisting}
\end{center}


\section{Задание 4}
Написать, по крайней мере, два варианта функции, которая возвращает
свой список аргумент без последнего элемента. 
\begin{center}	
	\begin{lstlisting}[label=d, caption={Выражение 4}]
	(defun without_last (lst)
		(cond
			((cdr lst) (cons (car lst) (without_last (cdr lst))))
			(T Nil)
		)
	)
	
	(defun without_last (lst)
		(reverse (cdr (reverse lst)))
	)
	
	\end{lstlisting}
\end{center}

\section{Задание 5}
Напишите функцию swap-first-last, которая переставляет в списке аргументе первый и последний элементы.
\begin{center}	
	\begin{lstlisting}[label=e, caption={Выражение 5}]
	(defun swap-first-last (lst)
		(append
			(last lst)
			(reverse (cdr (reverse (cdr lst))))
			(cons (car lst) Nil)
		)
	)
	\end{lstlisting}
\end{center}

\section{Задание 6}
Написать простой вариант игры в кости, в котором бросаются две
правильные кости. Если сумма выпавших очков равна 7 или 11 —
выигрыш, если выпало (1,1) или (6,6) — игрок имеет право снова
бросить кости, во всех остальных случаях ход переходит ко второму
игроку, но запоминается сумма выпавших очков. Если второй игрок не
выигрывает абсолютно, то выигрывает тот игрок, у которого больше
очков. Результат игры и значения выпавших костей выводить на экран с
помощью функции print.
\begin{center}	
	\begin{lstlisting}[label=f, caption={Выражение 6}]
(defvar dices)
(defvar roll1)
(defvar roll2)


(defun roll ()
    (+ (random 7) (random 7))
)

(defun is_win (throw)
    (cond
        ((or (= throw 7) (= throw 11)) T)
        (T Nil)
    )
)

(defun is_luck (throw)
    (cond
        ((or (= throw 2) (= throw 12)) T)
        (T Nil)
    )
)

(defun turn ()
    (terpri)
    (setq dices (roll))
    (princ "Rolled ")
    (princ dices)

    (cond
        ((is_win dices) 0)
        ((is_luck dices) (turn))
        (T dices)
    )
)

(defun game()
    (princ "First player is rolling: ")
    (setq roll1 (turn))
    (terpri)

    (cond
        ((eq roll1 0) (princ "First player won absolutely!"))
        (T ((lambda ()
                (princ "Second player is rolling: ")
                (setq roll2 (turn))
                (terpri)
                (cond
                    ((eq roll2 0) (princ "Second player won absolutely"))
                    ((eq roll1 roll2) (princ "Draw!"))
                    ((> roll1 roll2) (princ "First player won!"))
                    (T (princ "Second player won!"))
                )

        )))
    )
    (terpri)
)
	\end{lstlisting}
\end{center}

\section{Задание 7}
Написать функцию, которая по своему списку-аргументу lst определяет
является ли он палиндромом (то есть равны ли lst и (reverse lst)).
\begin{center}	
	\begin{lstlisting}[label=g, caption={Выражение 7}]
	(defun is_palindrom (lst)
		(equalp lst (reverse lst))
	)
	\end{lstlisting}
\end{center}

\section{Задание 8}
Напишите свои необходимые функции, которые обрабатывают таблицу из
4-х точечных пар:
(страна . столица), и возвращают по стране - столицу, а по столице —
страну
\begin{center}	
	\begin{lstlisting}[label=h, caption={Выражение 8}]
(defun show_on_map (lst item)
    (cond
        ((equal (caar lst) item) (cdar lst))
        ((equal (cdar lst) item) (caar lst))
        ((cdr lst) (show_on_map (cdr lst) item))
    )
)
	\end{lstlisting}
\end{center}

\section{Задание 9}
Напишите функцию, которая умножает на заданное число-аргумент
первый числовой элемент списка из заданного 3-х элементного спискааргумента, когда
a) все элементы списка --- числа,
6) элементы списка -- любые объекты.
\begin{center}	
	\begin{lstlisting}[label=i, caption={Выражение 9}]
		(defun mul_if_all_num (lst num)
		(cond
			((and
				(numberp num)
				(and
					(numberp (car lst))
					(and
						(numberp (cadr lst))
						(numberp (caddr lst))
					)
				)
			) (* (car lst) num))
			(T Nil)
		)
	)
	
	(defun mul_first_num (lst num)
		(cond
			((and (numberp (car lst)) (numberp num)) (* (car lst) num)))
			((cdr lst) (mul_first_num (cdr lst) num))
			(T Nil)
		)
	)
	\end{lstlisting}
\end{center}

