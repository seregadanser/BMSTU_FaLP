\chapter{Теоретическая часть}
\section{Базис языка}

Базис состоит из:
\begin{enumerate}
	\item структуры, атомы;
	\item встроенные (примитивные) функции (\texttt{atom}, \texttt{eq}, \texttt{cons}, \texttt{car}, \texttt{cdr});
	\item специальные функции и функционалы, управляющие обработкой структур, представляющих вычислимые выражения (\texttt{quote}, \texttt{cond}, \texttt{lambda}, \texttt{label}, \texttt{eval}).
\end{enumerate}

\section{Классификация функций}

Функции в \texttt{Lisp} классифицируют следующим образом:

\begin{itemize}
	\item чистые математические функции;
	\item рекурсивные функции;
	\item специальные функции --- формы (сегодня 2 аргумента, завтра - 5);
	\item псевдофункции (создают эффект на внешнем устройстве);
	\item функции с вариативными значениями, из которых выбирается 1;
	\item функции высших порядков --- функционал: используется для синтаксического управления программ (абстракция языка).
\end{itemize}

По назначению функции разделяются следующим образом:

\begin{itemize}
	\item конструкторы --- создают значение (\texttt{cons}, например);
	\item селекторы --- получают доступ по адресу (\texttt{car}, \texttt{cdr});
	\item предикаты --- возвращают \texttt{Nil}, \texttt{T}.
	\item функции сравнения -- такие как: \texttt{eq}, \texttt{eql}, \texttt{equal}, \texttt{equalp}.
\end{itemize}

\section{Способы создания функций}

Функции в \texttt{Lisp} можно задавать следующими способами:

\subsection*{\texttt{Lambda}-выражение}

Синтаксис:

\texttt{(lambda <$\lambda$-список> форма)}

Пример:

\begin{lstlisting} [
	float=h!,
	frame=single,
	numbers=left,
	abovecaptionskip=-5pt,
	caption={Функция определенная Lambda-выражением},
	label={cc:1},
	language={Lisp},
	]
	(lambda (a b) (sqrt (+ (* a a) (* b b))))
\end{lstlisting}

\subsection*{Именованная функция}

Синтаксис:

\texttt{(defun <имя функции> <$\lambda$-выражение>)}

Пример:

\begin{lstlisting} [
	float=h!,
	frame=single,
	numbers=left,
	abovecaptionskip=-5pt,
	caption={Пример определения именованной функции},
	label={cc:2},
	language={Lisp},
	]
	(defun hyp (a b) (sqrt (+ (* a a) (* b b))))
\end{lstlisting}

\section{Функции \texttt{car} и \texttt{cdr}}

\texttt{car} --- функция получения первого элемента точечной пары.

Примеры:

\begin{table}[!ht]
	\small
	\begin{center}
		\begin{tabular}{|c|c|c|}
			\hline
			\bfseries S-выражение & \bfseries Результат выполнения \texttt{car} \\\hline
			(A . B) & A \\\hline
			((A . B) . C) & (A . B) \\\hline
			A & ошибка \\\hline
		\end{tabular}
	\end{center}
\end{table}

\texttt{cdr} --- функция получения второго элемента точечной пары.

\begin{table}[!ht]
	\small
	\begin{center}
		\begin{tabular}{|c|c|c|}
			\hline
			\bfseries S-выражение & \bfseries Результат выполнения \texttt{cdr} \\\hline
			(A . B) & B \\ \hline
			(A . (B . C)) & (B . C) \\\hline
			A & ошибка \\\hline
		\end{tabular}
	\end{center}
\end{table}

\section{Функции \texttt{eq, eql, equal, equalp}}
(eq x y) является истиной тогда и только тогда, когда, x и y являются идентичными объектами. (В реализациях, x и y обычно равны eq тогда и только тогда, когда обращаются к одной ячейке памяти.)

Предикат eql истинен, если его аргументы равны eq, или если это числа одинакового типа и с одинаковыми значениями, или если это одинаковые буквы.

Предикат equal истинен, если его аргументы это структурно похожие (изоморфные) объекты. Грубое правило такое, что два объекта равны equal тогда и только тогда, когда одинаково их выводимое представление. Числа и буквы сравниваются также как и в eql. Символы сравниваются как в eq. Объекты, которые содержат другие элементы, будут равны equal, если они принадлежат одному типу и содержащиеся элементы равны equal.  

Два объекта равны equalp, если они равны equal, если они буквы и удовлетворяют предикату char-equal, который игнорирует регистр и другие атрибуты символов, если они числа и имеют одинаковое значение, даже если числа разных типов, если они включает в себя элементы, которые также равны equalp.
\begin{lstlisting}[label=a80, caption={Примеры}]
(eq 'a 'b) is false. 
(eq 'a 'a) is true. 
(eq 3 3) might be true or false, depending on the implementation. 
(eq 3 3.0) is false. 
(eq 3.0 3.0) might be true or false, depending on the implementation. 
(eq #c(3 -4) #c(3 -4)) 
might be true or false, depending on the implementation. 
(eq #c(3 -4.0) #c(3 -4)) is false. 
(eq (cons 'a 'b) (cons 'a 'c)) is false. 
(eq (cons 'a 'b) (cons 'a 'b)) is false. 
(eq '(a . b) '(a . b)) might be true or false. 
(progn (setq x (cons 'a 'b)) (eq x x)) is true. 
(progn (setq x '(a . b)) (eq x x)) is true. 
(eq #\A #\A) might be true or false, depending on the implementation. 
(eq "Foo" "Foo") might be true or false. 
(eq "Foo" (copy-seq "Foo")) is false. 
(eq "FOO" "foo") is false.


(eql 'a 'b) is false. 
(eql 'a 'a) is true. 
(eql 3 3) is true. 
(eql 3 3.0) is false. 
(eql 3.0 3.0) is true. 
(eql #c(3 -4) #c(3 -4)) is true. 
(eql #c(3 -4.0) #c(3 -4)) is false. 
(eql (cons 'a 'b) (cons 'a 'c)) is false. 
(eql (cons 'a 'b) (cons 'a 'b)) is false. 
(eql '(a . b) '(a . b)) might be true or false. 
(progn (setq x (cons 'a 'b)) (eql x x)) is true. 
(progn (setq x '(a . b)) (eql x x)) is true. 
(eql #\A #\A) is true. 
(eql "Foo" "Foo") might be true or false. 
(eql "Foo" (copy-seq "Foo")) is false. 
(eql "FOO" "foo") is false.


(equal 'a 'b) is false. 
(equal 'a 'a) is true. 
(equal 3 3) is true. 
(equal 3 3.0) is false. 
(equal 3.0 3.0) is true. 
(equal #c(3 -4) #c(3 -4)) is true. 
(equal #c(3 -4.0) #c(3 -4)) is false. 
(equal (cons 'a 'b) (cons 'a 'c)) is false. 
(equal (cons 'a 'b) (cons 'a 'b)) is true. 
(equal '(a . b) '(a . b)) is true. 
(progn (setq x (cons 'a 'b)) (equal x x)) is true. 
(progn (setq x '(a . b)) (equal x x)) is true. 
(equal #\A #\A) is true. 
(equal "Foo" "Foo") is true. 
(equal "Foo" (copy-seq "Foo")) is true. 
(equal "FOO" "foo") is false.
\end{lstlisting}


\section{Назначение и отличие \texttt{list} от \texttt{cons}}

\texttt{cons} --- функция конструирования точечной пары, на вход получает 2 значения и делает из них точечную пару.

\texttt{list} --- функция конструирования списка. На вход получает произвольное количество элементов и делает из них список.

Вызовы \texttt{(list 1 2 3 4)} и \texttt{(cons 1 (cons 2 (cons 3 (cons 4 Nil))))} эквивалентны, то есть дают одинаковый результат.

